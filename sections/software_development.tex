\begin{center}
    \begin{tabularx}{\textwidth}{X}
        {\large \textbf{SOFTWARE DEVELOPMENT}} \\
        \small \textbf{MicroSPAT: Microsatellite Parameterized Analysis Tools (\href{https://github.com/Greenhouse-Lab/MicroSPAT}{https://github.com/Greenhouse-Lab/MicroSPAT})} \\
        \small MicroSPAT is a collection of tools for semi-automated analysis of raw capillary electrophoresis (CE) data output by the ABI 3730. MicroSPAT integrates several features including a plate view for quality checking, automated ladder identification, sample based association of FSA data to keep data organized in a logical manner, automated allele identification using a clustering algorithm, automated artifact correction, automated quantification bias correction, and automated genotyping of samples with the option of manual curation. \\
        \small \textbf{SampleDB (\href{https://github.com/Greenhouse-Lab/sample_db}{https://github.com/Greenhouse-Lab/sample\_db})} \\
        \small SampleDB is an open source database implementation with graphical user interface to track clinical lab samples and their usage. SampleDB supports tracking of sample specimens by barcoded tubes, for example the Matrix \textsuperscript{TM} barcoded tube system. \\
        \small \textbf{Sketch to Diagram (\href{https://github.com/m-murphy/sketch-to-diagram}{https://github.com/m-murphy/sketch-to-diagram})} \\
        \small Sketch to Diagram was built in a hackathon type event applying modern machine learning methods. SKtD is a proof of concept web app that uses a convolutional neural network based model to identify user drawings which are then converted into rendered objects. The goal of SKtD was to demonstrate how a non-trivial problem such as gesture recognition could be tackled in a flexible manner using modern machine learning in a user application. The application runs completely within the web browser at \href{http://sketch-to-diagram.surge.sh}{http://sketch-to-diagram.surge.sh}. \\
    \end{tabularx}
\end{center}